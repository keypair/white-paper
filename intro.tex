\section{Introduction}

\Aet\ \cite{AE,UlfWigerCodeMesh2018} is an open source blockchain
platform for advanced smart contract applications
that can be used by millions of users. Several key technologies
are put in place to achieve these scaling requirements: state channels, the
next generation of Nakamoto consensus algorithm (Bitcoin-NG) and the
efficient FATE virtual machine for smart contract execution.

Increases in transaction throughput, possibly millions of
transactions per second, can be achieved via state channels (Sect.\
\ref{sect:channels}),
an off-chain encrypted peer-to-peer communication protocol for trustless
execution of smart contracts. After
agreeing on-chain to collaborate in a state channel, parties communicate
mutually authenticated transactions to each other. These transactions don't
have to be recorded on chain and thus can be exchanged at much higher
speed. Closing the channel, with or without dispute resolution, is performed
on-chain again.

Further increase of throughput is achieved by using a novel consensus
algorithm.
The platform runs as a decentralized trustless distributed ledger using
Proof-of-Work (PoW)~\cite{dwork1992pricing,back1997hashcash,Tromp2015CuckooCA}
for leader election.
In Sect.~\ref{sect:mining} we explain how we improve the scalability of
original Nakamoto consensus~\cite{SN} by leveraging
Bitcoin-NG~\cite{Eyal:2016:BSB:2930611.2930615}. The result of this change
is a throughput of about 117 on-chain transactions per
second~\cite{source?} with a (micro)blocktime of three seconds and with
low latency as opposed to the seven transactions per second of Bitcoin at ten
minutes block confirmation time.

Additionally to the PoW, a weighted delegated governance system has been
implemented. This way the users of \blockchain\ can express their
(hardfork)choice in a tamperproof-way, signalling to the miners which fork
would be more attractive in the future.

The \blockchain\ offers a variety of transaction types
based on common applications on other blockchain implementations.
For example, many blockchain users want to assign human readable, persistent
names to objects on chain. The \blockchain\ provides a
set of transactions that make this easy for developers, without the need to
implement a smart contract for it (\ref{sect:aens}). Another example is a set
of transactions to register and query oracles (\ref{sect:aeoracle}), which
provide data from outside the blockchain.
These transactions are explained in more detail in
Sect.~\ref{sect:transactions}.

Many features yet to be invented can already be implemented
by users if they use the \aet\ smart contract language
\textit{Sophia}. Sophia is a Turing-complete functional language
designed with security in mind. Many mistakes that one can make in
other contract languages are impossible to make or are easier to detect
when using Sophia. In Sect.\ \ref{sect:sophia} we present key
ideas of the language.

Contracts are compiled to bytecode, which is executed on the highly
efficient virtual machine \textit{FATE}. Similar to other smart contract
languages every operation has a \enquote{gas cost} associated to it. This cost
reflects
the amount of work needed to execute a contract. The FATE virtual machine is
specifically designed for
\aet\ to meet high security and efficiency demands, which we
explain in more detail in Sect.~\ref{sect:fate}.

The reference implementation of the \aet\ protocol is written using the
functional language Erlang~\cite{Armstrong:2010:ERL:1810891.1810910}. This
language originates from the telecommunication industry and is used in large
distributed and highly concurrent systems (e.g.\ WhatsApp). However, the choice
of implementation language has no further implications for the techniques used
and described in this paper.
