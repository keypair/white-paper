\section{Future ambitions and past evolution}

The \blockchain\ went live on November 28th, 2018. For the curious reader,
there is a timestamp in each block and the first mined key-block has timestamp
``1543373685748'', which is the time in milliseconds using POSIX time. Since
that first date, 3 major protocol updates have successfully be applied as of
March 2020,
enriching the \blockchain\ with new features. The new protocols are effective
at a certain height and the software supports the old protocol under that
height and the new protocol from that height. Each protocol is referred to by
name for ease in communication with developers of blockchain applications:
\textit{Roma}, effective at height 0, \textit{Minerva}, effective at height
47800, \textit{Fortuna}, effective at height 90800, and \textit{Lima},
effective at height 161150.

In the future, there will be more protocol upgrades with additional features,
some of which we are going to outline in this section.

\subsection{Formal verification}

Over the course of its existence Ethereum has had many flaws of deployed
contracts uncovered and abused. Some of these were simple developer errors and
others very subtle due to the complexity of both the EVM and Solidity.

Formal verification is one of the approaches to prevent these problems by
allowing code to be proven correctly with regards to a given specification.
Formal verification is used in many areas where high assurance is of vital
importance, e.g. cryptographic systems.
Take the parity multi-sig contract flaw, which allowed anyone to destroy one of
the libraries used by the multi-sig contract rendering it useless, as an
example. A formal specification of this contract could include the assumption
that destruction functions can only be called by authorized accounts or not at
all. Checking this specification against the code would then have raised an
error. But there is certainly still the problem of writing good specifications.

Since Sophia was written with formal verification in mind, we have laid most of
the ground work required to provide these tools. Together with Sophia being a
functional language, this should prevent many classes of bugs plaguing Solidity
smart contracts.

\subsection{Native tokens}

The ERC-20 standard, which specified an interface for fungible tokens, was
arguably one of the biggest drivers of adoption for Ethereum. With that came
the ability for anyone to design and test economic systems, which was a great
catalyst for the innovation happening on Ethereum. A big drawback of the token
contracts deployed was and still is the requirement to pay gas and thus own
Eth, which can be a major hurdle for users, who don't necessarily want to own
any Eth or even know what that is. In addition to that, interacting or
integrating with token can be a pain, especially since the interface evolved
over time and if upgradability was not baked into the contract there are only
very clumsy upgrade paths for old contracts.

To address these drawbacks, \aet\ will make tokens native to the blockchain.
That opens the way to be able to use tokens to pay transaction fees, which
frees users from the requirement to own any other tokens. It will make usage
cheaper since the basic token logic and storage can be optimized in the virtual
machine. Finally, this will allow tokens to benefit from updates and upgrades
for free, without needing complex upgrade strategies.

\subsection{Computational integrity}

Computational integrity assures that a given function was computed correctly.
This could mean that a coin transfer correctly deducts coins from one account
while adding it to another or the correct execution of a smart contract call.
Currently assuring these correct executions is done by miners and node
operators via complex consensus rules. It also requires everyone who wants to
check the correct execution to re-run the full computation, which could be very
costly.

There are different ways that allow checking the integrity of a function
execution but we are going to assume that the prover, the agent running the
computation and trying to prove that they executed it correctly, can generate a
succinct proof. Succinctness implies that checking the proof is
computationally less expensive than actually re-running the computation. This
proof can then be read and validated by another party, the verifier, who wants
to check the integrity of the computation.

The existence of such primitives then allows scaling, since checking the
proof takes less resources than running the computation. Additionally, these
proofs can be generated in such a way that the function itself can be private,
while still being verifiable by a third party, which would be a big gain in
privacy.

Sophia already comes with some primitives to write efficient proof verifiers
and in the future we want to integrate these primitives even further into the
\aet\ protocol, to make it faster and private.

\subsection{Further scaling}

The Bitcoin-NG consensus described in section \ref{sect:mining} allows us to
handle around 120 transaction per second which is sufficient to process
everything almost without delay or block congestion at the time of writing.
But in a future where millions of people want to use \aet\ we will need higher
throughput. One partial solution, state channels, are already available today
and will become more relevant as usability improves.

%Besides state channels exist many other different approaches, which can be used
%alongside. The most obvious solution is to improve the consensus algorithm in
%such a way that it can handle higher throughput and there are already many
%other consensus algorithms, which can improve on Bitcoin-NG.

Another direction to go is to split the blockchain into distinct parts, which
is also common for databases. These parts are then called shards and each one
will then be responsible for only a subset of all available transactions.
Dividing up the work like this would then offer each shard more room to scale
just by virtue of only having to handle a fraction of the previous load. Shards
will still have to communicate with each other, which is a possible bottleneck,
and have to know of each others existence. But overall the system could present
a big gain in throughput.

The third widely researched concept are then child-chains, app-chains or
side-chains. Other names for child-chains are app-chains or side-chains.
In the context of aeternity they will (most likely) be called hyperchains.
Unlike sharding, side-chains do not divide a global state space but each
one has its own state. The idea of this approach is to have many specific,
maybe even single purpose, chains which do not necessarily have to know of each
other. The name child-chain implies that they are
hooked into a parent chain. Communication between childen and parents, accross
any hierarchy usually can happen via cross-chain atomic swaps. Alternatively it
can be achieved via some sort of hierarchical escalation, in similar fashion to
state channel force progress.
Childchain then imply that each use case or æpp could get its own chain,
thus only having to handle transactions for this one æpp and not be concerned
with all the other applications. This in turn would then allow each æpp much
more throughput and maybe even specific optimizations.

All the solutions described here offer different trade-offs but could certainly
be used in tandem. \aet\ will most likely evolve into all the presented
approaches in order to meet demands by billions of users.



\subsection{Differences to v0.1 \blockchain\ whitepaper}
The \blockchain\ has evolved from a whitepaper published in 2017 to a working
system loaded with real value, also known as the \aet\ mainnet. In the process
several things got adapted and fully re-engineered from scratch by (co)creators
of Erlang, Haskell, Skala and Agda, languages in industrial use today.

- Next generation Nakamoto consensus (also with cuckoo cycle PoW)

- more state on chain.

- secure, functional, higher level, smarter contract language.

- more advanced and performat virtual machine.

- integration of higher level concepts into smart contract language.

- differentiation between oracles and prediction markets.

- accounts are now public keys and not anymore IDs.

